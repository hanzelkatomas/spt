\hypertarget{index_main}{}\section{Popis programu}\label{index_main}
Program má sloužit k převodu čísel v dekadickém formátu na čísla v dvojkové, osmičkové a šestnáckové soustavě
\begin{DoxyItemize}
\item Uživatel zadá číslo v dekadickém formátu
\item Vybere danou soustavu, do které chce číslo převést
\item V případě potřeby může pokračovat v procesu
\item Vstupy jsou ošetřené proti pádu programu
\item Velikost pole se vytváří na základě výpočtu ze vstupu uživatele (vytvoří se přesně taková velikost jak je minimálně potřeba)
\end{DoxyItemize}\hypertarget{index_variables}{}\section{Proměnné programu}\label{index_variables}

\begin{DoxyItemize}
\item int x -\/ input uživatel čísla, které chce převést
\item string pokracovat -\/ slouží k uchování hodnoty, zda chce uživatel pokračovat v převodu
\item bool ok slouží k ověření, že uživatel zadal validní vstup
\item string soustava -\/ slouží k uložení volby do jaké soustavy chce uživatel převést číslo
\item double pre\+Velikost\+Bin/\+Osm/\+Hex -\/ slouží k uložení výpočtu pro přesnou velikost pole
\item int hodnota\+Bin/\+Osm/\+Hex -\/ slouží k počítání v převodním algoritmu
\item int Bin/\+Osm/\+Hex\+Velikost\+Pole -\/ slouží k přetypování proměnné z double na int a zároveň funguje jako informace pro cyklus kdy má skončit
\item pole char\+Array\+Bin/\+Osm/\+Hex -\/ slouží k uložení samotného výsledku výpočtu
\end{DoxyItemize}\hypertarget{index_methods}{}\section{Použité metody}\label{index_methods}

\begin{DoxyItemize}
\item Math.\+Log -\/ slouží k vypočítání ideální velikost pole. Logaritmus o základu 2 sloužil k počítání binárního převodu. Základ 8 k osmičkovému převodu atp.
\item Math.\+Floor -\/ zaokrouhlení celého čísla směrem dolů, číslo se použilo pro vytvoření ideální velikosti pole.
\item Array.\+Reverse -\/ slouží k \char`\"{}přehození\char`\"{} celého pole. Jelikož je výsledek z počítání zapsán naopak, stejně jak když provádíme výpočet na papíře. 
\end{DoxyItemize}